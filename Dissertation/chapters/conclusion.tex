% Conclusion (?)
% 	Make it clear in first paragraph what your project was
% 		about and how well you've done it
% 	Will fill in marking sheet after they have done it.
% 	Remind them how awesome project is
% 	If did again, might do... XYZ
% * Refer back to Introduction.
% * How would have planned the project if starting again with benefit of hindsight

% ************
% Should have: 
% 	intro
% 	content
% 	summary
% ************

\chapter{Conclusion}
In my project I created a search engine to be run by independent and distributed online social networks. The search engine honours the fundamental ideal of independent online social networks of allowing their users control over what and how much data about them is made publicly available, and also solves the problem of allowing users of independent online distributed social networks to find and reconnect with their friends regardless of in which online social network they have their profile.

I built this service on top of Distributed Hash Tables and invented a scheme for enabling fuzzy and predictive searches across key-value stores. This scheme works well in its current form for installations with small to medium sized data sets, and could work well in larger installations if slightly tweaked.

As part of my project I implemented two Distributed Hash Tables. While both work correctly, my implementation of Pastry works significantly better than Chord, and gives quite impressive results.

My project was highly successful as a proof of concept, but would need further refinement before being deployed by real social networks, which is in line with what I set out to do. 

\mbox{}

While being a success in its own right, the project has also been an excellent learning experience.
I had never used the programming language Erlang for anything but trivial ``hello world" applications before the start of my project and now have a solid working knowledge of it and its libraries. I also got significant exposure to working with highly distributed systems and deploying software across up to 100 machines.
The same project implemented again from the beginning would cost me significantly less time and effort considering all the knowledge I now possess, but that itself can also be considered a success, considering knowledge acquisition is a central component of what role I believe the Part II projects are designed to achieve.

\mbox{}

As a whole the project solves a very real problem that needed addressing to help independent distributed social networks stand a fighting chance in gaining ground against the social networking gorilla Facebook. There is still a lot of work that needs doing, but my project serves an important role in highlighting an obstacle that needs addressing, and proposed a solution for how it can be tackled.
