% * motivation for the project
% * how fits into the broad area of surrounding CS
% * brief survey of previous related work
% * should get to know what the project is about by reading intro
% Paragraph by paragraph
% 	My project is about this
% 	I set out to do this 
% 	I did this

% ************
% Should have: 
% 	intro
% 	content
% 	summary
% ************

% Allowed to have 1200 words.

\chapter{Introduction}
In the last few years, multiple independent distributed online social networks have been made with the explicit goal of breaking Facebook's monopoly on social networking services and empower users by giving them ownership over their own data.
While I applaud these initiatives, I also see a shortcoming they all have in common. Facebook makes it easy to find one's friends through search, whereas independent online social networks, often spanning multiple installations and providers, require their users to know where and in which online social network installation their friends have profiles. Not only does this make for a bad user experience, it also limits the potential these services might have to reach wider audiences and greater success.

I created a search engine that honours the ideals of the independent online social networks of giving their users control over what and how much data is made publicly available, yet at the same time letting them find and reconnect with their friends regardless of in which online social network they decide to host their profiles.

I built a proof of concept distributed search engine allowing predictive searches and fuzzy matching for misspelled and incomplete names. This search engine was built on top of Distributed Hash Tables. Each installation of an independent distributed online social network runs a copy of my software, and together make a search network available to all users. The cost of running the search engine is then distributed between the online social networks using it, and there would be no central authority with complete control over the data. Additionally, the independent online social network installations, or even individual users, could themselves decide how much and what data is made available.

I made working implementations of both the Distributed Hash Tables Chord and Pastry. While my implementation of Chord works sufficiently well, my implementation of Pastry is of very high performance.

Small scale testing also shows that the search network works as expected, providing user-friendly predictive searches and showing profile images alongside search results that are updating as the user types the search query.

I also created a web application allowing me to control the network of search nodes remotely, as well as initiate experimental runs and collect experimental data for my project evaluation.
