% The master copy of this demo dissertation is held on my filespace
% on the cl file serve (/homes/mr/teaching/demodissert/)

% Last updated by MR on 2 August 2001

\documentclass[12pt,twoside,notitlepage]{report}

\usepackage{a4}
\usepackage{verbatim}

\input{epsf}                            % to allow postscript inclusions
% On thor and CUS read top of file:
%     /opt/TeX/lib/texmf/tex/dvips/epsf.sty
% On CL machines read:
%     /usr/lib/tex/macros/dvips/epsf.tex



\raggedbottom                           % try to avoid widows and orphans
\sloppy
\clubpenalty1000%
\widowpenalty1000%

\addtolength{\oddsidemargin}{6mm}       % adjust margins
\addtolength{\evensidemargin}{-8mm}

\renewcommand{\baselinestretch}{1.1}    % adjust line spacing to make
                                        % more readable

\begin{document}

\bibliographystyle{plain}


%%%%%%%%%%%%%%%%%%%%%%%%%%%%%%%%%%%%%%%%%%%%%%%%%%%%%%%%%%%%%%%%%%%%%%%%
% Title


\pagestyle{empty}

\hfill{\LARGE \bf Martin Richards}

\vspace*{60mm}
\begin{center}
\Huge
{\bf How to write a dissertation in \LaTeX} \\
\vspace*{5mm}
Diploma in Computer Science \\
\vspace*{5mm}
St John's College \\
\vspace*{5mm}
\today  % today's date
\end{center}

\cleardoublepage

%%%%%%%%%%%%%%%%%%%%%%%%%%%%%%%%%%%%%%%%%%%%%%%%%%%%%%%%%%%%%%%%%%%%%%%%%%%%%%
% Proforma, table of contents and list of figures

\setcounter{page}{1}
\pagenumbering{roman}
\pagestyle{plain}

\chapter*{Proforma}

{\large
\begin{tabular}{ll}
Name:               & \bf Martin Richards                       \\
College:            & \bf St John's College                     \\
Project Title:      & \bf How to write a dissertation in \LaTeX \\
Examination:        & \bf Diploma in Computer Science, July 2001        \\
Word Count:         & \bf 1587\footnotemark[1]
(well less than the 12000 limit) \\
Project Originator: & Dr M.~Richards                    \\
Supervisor:         & Dr M.~Richards                    \\ 
\end{tabular}
}
\footnotetext[1]{This word count was computed
by {\tt detex diss.tex | tr -cd '0-9A-Za-z $\tt\backslash$n' | wc -w}
}
\stepcounter{footnote}


\section*{Original Aims of the Project}

To write a demonstration dissertation\footnote{A normal footnote without the
complication of being in a table.} using \LaTeX\ to save
student's time when writing their own dissertations. The dissertation
should illustrate how to use the more common \LaTeX\ constructs. It
should include pictures and diagrams to show how these can be
incorporated into the dissertation.  It should contain the entire
\LaTeX\ source of the dissertation and the Makefile.  It should
explain how to construct an MSDOS disk of the dissertation in
Postscript format that can be used by the book shop for printing, and,
finally, it should have the prescribed layout and format of a diploma
dissertation.


\section*{Work Completed}

All that has been completed appears in this dissertation.

\section*{Special Difficulties}

Learning how to incorporate encapulated postscript into a \LaTeX\
document on both CUS and Thor.
 
\newpage
\section*{Declaration}

I, [Name] of [College], being a candidate for Part II of the Computer
Science Tripos [or the Diploma in Computer Science], hereby declare
that this dissertation and the work described in it are my own work,
unaided except as may be specified below, and that the dissertation
does not contain material that has already been used to any substantial
extent for a comparable purpose.

\bigskip
\leftline{Signed [signature]}

\medskip
\leftline{Date [date]}

\cleardoublepage

\tableofcontents

\listoffigures

\newpage
\section*{Acknowledgements}

This document owes much to an earlier version written by Simon Moore
\cite{moore95}.  His help, encouragement and advice was greatly 
appreciated.

%%%%%%%%%%%%%%%%%%%%%%%%%%%%%%%%%%%%%%%%%%%%%%%%%%%%%%%%%%%%%%%%%%%%%%%
% now for the chapters

\cleardoublepage        % just to make sure before the page numbering
                        % is changed

\setcounter{page}{1}
\pagenumbering{arabic}
\pagestyle{headings}

\chapter{Introduction}

\section{Overview of the files}

This document consists of the following files:

\begin{itemize}
\item {\tt Makefile} --- The Makefile for the dissertation and Project Proposal
\item {\tt diss.tex} --- The dissertation
\item {\tt propbody.tex} --- Appendix~C  -- the project proposal 
\item {\tt proposal.tex}  --- A \LaTeX\ main file for the proposal 
\item{\tt figs} -- A directory containing diagrams and pictures
\item{\tt refs.bib} --- The bibliography database
\end{itemize}

\section{Building the document}

This document was produced using \LaTeXe which is based upon
\LaTeX\cite{Lamport86}.  To build the document you first need to
generate {\tt diss.aux} which, amongst other things, contains the
references used.  This if done by executing the command:

{\tt latex diss}

\noindent
Then the bibliography can be generated from {\tt refs.bib} using:

{\tt bibtex diss}

\noindent
Finally, to ensure all the page numbering is correct run {\tt latex}
on {\tt diss.tex} until the {\tt .aux} files do not change.  This
usually takes 2 more runs.

\subsection{The makefile}

To simplify the calls to {\tt latex} and {\tt bibtex}, 
a makefile has been provided, see Appendix~\ref{makefile}. 
It provides the following facilities:

\begin{itemize}

\item{\tt make} \\
 Display help information.

\item{\tt make prop} \\
 Run {\tt latex proposal; xdvi proposal.dvi}.

\item{\tt make diss.ps} \\
 Make the file {\tt diss.ps}.

\item{\tt make gv} \\
 View the dissertation using ghostview after performing 
{\tt make diss.ps}, if necessary.

\item{\tt make gs} \\
 View the dissertation using ghostscript after performing 
{\tt make diss.ps}, if necessary.

\item{\tt make count} \\
Display an estimate of the word count.

\item{\tt make all} \\
Construct {\tt proposal.dvi} and {\tt diss.ps}.

\item{\tt make pub} \\ Make a {\tt .tar} version of the {\tt demodiss}
directory and place it in my {\tt public\_html} directory.

\item{\tt make clean} \\ Delete all files except the source files of
the dissertation. All these deleted files can be reconstructed by
typing {\tt make all}.

\item{\tt make pr} \\
Print the dissertation on your default printer.

\end{itemize}


\section{Counting words}

An approximate word count of the body of the dissertation may be
obtained using:

{\tt wc diss.tex}

\noindent
Alternatively, try something like:

\verb/detex diss.tex | tr -cd '0-9A-Z a-z\n' | wc -w/




\cleardoublepage



\chapter{Preparation}

This chapter is empty!


\cleardoublepage
\chapter{Implementation}

\section{Verbatim text}

Verbatim text can be included using \verb|\begin{verbatim}| and
\verb|\end{verbatim}|. I normally use a slightly smaller font and
often squeeze the lines a little closer together, as in:

{\renewcommand{\baselinestretch}{0.8}\small\begin{verbatim}
GET "libhdr"
 
GLOBAL { count:200; all  }
 
LET try(ld, row, rd) BE TEST row=all
                        THEN count := count + 1
                        ELSE { LET poss = all & ~(ld | row | rd)
                               UNTIL poss=0 DO
                               { LET p = poss & -poss
                                 poss := poss - p
                                 try(ld+p << 1, row+p, rd+p >> 1)
                               }
                             }
LET start() = VALOF
{ all := 1
  FOR i = 1 TO 12 DO
  { count := 0
    try(0, 0, 0)
    writef("Number of solutions to %i2-queens is %i5*n", i, count)
    all := 2*all + 1
  }
  RESULTIS 0
}
\end{verbatim}
}

\section{Tables}

\begin{samepage}
Here is a simple example\footnote{A footnote} of a table.

\begin{center}
\begin{tabular}{l|c|r}
Left      & Centred & Right \\
Justified &         & Justified \\[3mm]
%\hline\\%[-2mm]
First     & A       & XXX \\
Second    & AA      & XX  \\
Last      & AAA     & X   \\
\end{tabular}
\end{center}

\noindent
There is another example table in the proforma.
\end{samepage}

\section{Simple diagrams}

Simple diagrams can be written directly in \LaTeX.  For example, see
figure~\ref{latexpic1} on page~\pageref{latexpic1} and see
figure~\ref{latexpic2} on page~\pageref{latexpic2}.

\begin{figure}
\setlength{\unitlength}{1mm}
\begin{center}
\begin{picture}(125,100)
\put(0,80){\framebox(50,10){AAA}}
\put(0,60){\framebox(50,10){BBB}}
\put(0,40){\framebox(50,10){CCC}}
\put(0,20){\framebox(50,10){DDD}}
\put(0,00){\framebox(50,10){EEE}}

\put(75,80){\framebox(50,10){XXX}}
\put(75,60){\framebox(50,10){YYY}}
\put(75,40){\framebox(50,10){ZZZ}}

\put(25,80){\vector(0,-1){10}}
\put(25,60){\vector(0,-1){10}}
\put(25,50){\vector(0,1){10}}
\put(25,40){\vector(0,-1){10}}
\put(25,20){\vector(0,-1){10}}

\put(100,80){\vector(0,-1){10}}
\put(100,70){\vector(0,1){10}}
\put(100,60){\vector(0,-1){10}}
\put(100,50){\vector(0,1){10}}

\put(50,65){\vector(1,0){25}}
\put(75,65){\vector(-1,0){25}}
\end{picture}
\end{center}
\caption{\label{latexpic1}A picture composed of boxes and vectors.}
\end{figure}

\begin{figure}
\setlength{\unitlength}{1mm}
\begin{center}

\begin{picture}(100,70)
\put(47,65){\circle{10}}
\put(45,64){abc}

\put(37,45){\circle{10}}
\put(37,51){\line(1,1){7}}
\put(35,44){def}

\put(57,25){\circle{10}}
\put(57,31){\line(-1,3){9}}
\put(57,31){\line(-3,2){15}}
\put(55,24){ghi}

\put(32,0){\framebox(10,10){A}}
\put(52,0){\framebox(10,10){B}}
\put(37,12){\line(0,1){26}}
\put(37,12){\line(2,1){15}}
\put(57,12){\line(0,2){6}}
\end{picture}

\end{center}
\caption{\label{latexpic2}A diagram composed of circles, lines and boxes.}
\end{figure}



\section{Adding more complicated graphics}

The use of \LaTeX\ format can be tedious and it is often better to use
encapsulated postscript to represent complicated graphics.
Figure~\ref{epsfig} and ~\ref{xfig} on page \pageref{xfig} are
examples. The second figure was drawn using {\tt xfig} and exported in
{\tt.eps} format. This is my recommended way of drawing all diagrams.


\begin{figure}[tbh]
\centerline{\epsfbox{figs/cuarms.eps}}
\caption{\label{epsfig}Example figure using encapsulated postscript}
\end{figure}

\begin{figure}[tbh]
\vspace{4in}
\caption{\label{pastedfig}Example figure where a picture can be pasted in}
\end{figure}


\begin{figure}[tbh]
\centerline{\epsfbox{figs/diagram.eps}}
\caption{\label{xfig}Example diagram drawn using {\tt xfig}}
\end{figure}




\cleardoublepage
\chapter{Evaluation}

\section{Printing and binding}

If you have access to a laser printer that can print on two sides, you
can use it to print two copies of your dissertation and then get them
bound by the Computer Laboratory Bookshop. Otherwise, print your
dissertation single sided and get the Bookshop to copy and bind it double
sided.


Better printing quality can sometimes be obtained by giving the
Bookshop an MSDOS 1.44~Mbyte 3.5" floppy disc containing the
Postscript form of your dissertation. If the file is too large a
compressed version with {\tt zip} but not {\tt gnuzip} nor {\tt
compress} is acceptable. However they prefer the uncompressed form if
possible. From my experience I do not recommend this method.

\subsection{Things to note}

\begin{itemize}
\item Ensure that there are the correct number of blank pages inserted
so that each double sided page has a front and a back.  So, for
example, the title page must be followed by an absolutely blank page
(not even a page number).

\item Submitted postscript introduces more potential problems.
Therefore you must either allow two iterations of the binding process
(once in a digital form, falling back to a second, paper, submission if
necessary) or submit both paper and electronic versions.

\item There may be unexpected problems with fonts.

\end{itemize}

\section{Further information}

See the Computer Lab's world wide web pages at URL:

{\tt http://www.cl.cam.ac.uk/TeXdoc/TeXdocs.html}


\cleardoublepage
\chapter{Conclusion}

I hope that this rough guide to writing a dissertation is \LaTeX\ has
been helpful and saved you time.




\cleardoublepage

%%%%%%%%%%%%%%%%%%%%%%%%%%%%%%%%%%%%%%%%%%%%%%%%%%%%%%%%%%%%%%%%%%%%%
% the bibliography

\addcontentsline{toc}{chapter}{Bibliography}
\bibliography{refs}
\cleardoublepage

%%%%%%%%%%%%%%%%%%%%%%%%%%%%%%%%%%%%%%%%%%%%%%%%%%%%%%%%%%%%%%%%%%%%%
% the appendices
\appendix

\chapter{Latex source}

\section{diss.tex}
{\scriptsize\verbatiminput{diss.tex}}

\section{proposal.tex}
{\scriptsize\verbatiminput{proposal.tex}}

\section{propbody.tex}
{\scriptsize\verbatiminput{propbody.tex}}



\cleardoublepage

\chapter{Makefile}

\section{\label{makefile}Makefile}
{\scriptsize\verbatiminput{makefile.txt}}

\section{refs.bib}
{\scriptsize\verbatiminput{refs.bib}}


\cleardoublepage

\chapter{Project Proposal}


% Draft #1 (final?)

\vfil

\centerline{\Large Friend search for distributed social networks }
\vspace{0.4in}
\centerline{\large Sebastian Probst Eide, St Edmunds College }
\vspace{0.3in}
\centerline{\large Originator: Sebastian Probst Eide}
\vspace{0.3in}
\centerline{\large 4 October 2010}

\vfil

\subsection*{Special Resources Required}
Personal laptop for development and initial testing (1.86 Ghz, 2GB Ram) \\
CL machine with the Erlang VM installed as a backup \\
Virtual Server infrastructure, the like of Amazon EC2, for parts of the evaluation \\
\vspace{0.2in}

\noindent
{\bf Project Supervisors:} Dr David Eyers and Dr David Evans
\vspace{0.2in}

\noindent
{\bf Director of Studies:} Dr Robert Harle
\vspace{0.2in}
\noindent
 
\noindent
{\bf Project Overseers:} Unknown ********** PLEASE UPDATE! ***********

\vfil
\pagebreak

% Main document

\section*{Introduction}

It is hard to get started using independent, distributed online social networks as it is frequently difficult to find and connect with your existing friends, not knowing in which social networking system(s) they host their profiles and where those networks are hosted. The purpose of this project is to lay the foundation for a decentralised and distributed friend search engine that can be offered alongside installations of independent social networks allowing users to easily reconstruct their social graph in the online social network(s) of their choice. The focus of the project will be on the data storage layer of the search engine. I will compare and contrast different Distributed Hash Tables that I implement in Erlang. Erlang is chosen because it is known to be well suited for developing concurrent and distributed systems.

A front end, allowing basic searches to be performed, will also be created, but mainly to provide a way to rapidly exercise the data storage layer. More user-oriented functionality needed to allow the project to be used on a larger scale will be left out so as to limit the scope of the project.

\section*{Work that has to be done}

There are a number of distributed hash table designs available.\footnote{ Wikipedia currently lists 8 different protocols } I have decided to implement and compare the following three, chosen because they are well documented algorithms but importantly differ in the way they perform their routing: Chord, Kademilia and Pastry. As an example in how they differ, consider how Pastry allows for heuristics based on anything from ping to available bandwidth or combinations thereof, Kademilia uses XOR arithmetic to determine the distance between nodes as a routing heuristic, while Chord has none of the above, using simplicity as its strength. These different approaches to the same problem make the algorithms excellent candidates to compare.

The main parts of the project are to:

\begin{itemize}
  \item Implement the data storage layer of the search engine in Chord, Kademilia and Pastry using a uniform API that allows the system to use any one of the three without additional changes
  
  \item Implement infrastructure that facilitates testing and monitoring of the system. More specifically it should allow:

  \begin{itemize}
    \item Starting and stopping virtual servers across the different service providers to minimize server rental costs between testing sessions. The servers will be used to test different aspects of the distributed hash tables
    \item Start and stop search nodes across the physical servers
    \item Display how many search nodes are available in the system, and potentially some metric for how they are interconnected in terms of latency and bandwidth
    \item Add and remove test data from the system. The system will be tested with \emph{Database of names} from Facebook which I have access to. It is of significant size and importantly, contain keys with non-random distributions making for a more realistic dataset
    \item Perform repeatable load testing on the system where tests as an example could compare read/write performance for different key and value sizes and different numbers of key-value pairs
    \item Count the number of jumps and the time a key lookup needs in order to find a data item, in addition other appropriate descriptive statistical measures for writes and lookups in fixed key-value datasets
    \item Being able to eliminate and add subsets of storage nodes in a repeatable fashion to test how the different Distributed Hash Tables cope with nodes disappearing and appearing
  \end{itemize}

  \item Setup a Linux image that can be run across Infrastructure as a Service providers\footnote{Vagrant seems like a likely candidate to help automate this (http://vagrantup.com)}

  \item Implement a web front end to allow users to perform basic searches across the data storage layer

\end{itemize}

Please note that while the following aspects of the search server are secondary to the project and will not initially be implemented, they all, should time permit, serve as excellent project extensions:

\begin{itemize}
  \item Fuzzy searches allowing the user to misspell names. 
  \item Support composite keys to allow searching for different attributes of a record
  \item Predictive searches
  \item Searches taking knowledge about social circles from online social networks, or similar metadata, into account in order to more intelligently prioritise and order the search results returned to the user
  \item Protections against malicious use of the storage network like broadcasting data, attempting to overload nodes with requests or using the network to store spam or pollute the namespace with spammy records
\end{itemize}


\section*{Starting Point}

I have a reasonable working knowledge of Erlang and Linux and development of web based systems. The algorithms that will be implemented have all previously been implemented in other languages, and are used in production systems, so finding information about them should be possible. I have not yet used Amazon EC2 or any of the other Infrastructure as a Service (IaaS) providers that could be used to perform testing on the system in a distributed manner.


\section*{Success criterion}

I regard the project as successful if I have working implementations of the three Distributed Hash Tables that allow me to set and retrieve values based on keys across a distributed network of machines, and also metrics for how the performance of key-lookup varies by node-, key-count and distributed hash table type and recommendations for future work based on the metrics collected.

The search component of the project, which is the application part that uses the distributed hash tables as a datastore to allow users to find their friends, has been left out of the success criterion due to it being harder to quantify and evaluate well.

\section*{Difficulties to Overcome}

The following main learning tasks will have to be undertaken before 
the project can be started:

\begin{itemize}

\item To learn and fully understand the Chord, Kademilia and Pastry algorithms.

\item To learn how to do network communication in Erlang other than the built in message passing. I prefer the individual nodes to communicate over TCP or UDP as that frees the design from assumptions regarding the language of implementation, and also removes the substantial overhead incurred by having each Erlang VM keep track of the state and availability of all the other nodes network wide, which would be in stark contrast with the distributed hash tables need to only maintain rather minimalistic routing tables. Additionally there are security issues when allowing Erlang VMs to connect directly in untrusted networks as any node is allowed to execute arbitrary code on any other connected node

\item To find a way to test the system on geographically distributed nodes.

\end{itemize}


\section*{Resources}

Some aspects of this project (amongst others the heuristics in Pastry that take locality into account when routing) are more interesting to test in nodes that are geographically distributed. For this reason using server instances from a provider like Amazon AWS or PlanetLab seem like a good idea. Ways of getting access to time on such infrastructure for academic purposes is currently being looked into. For the majority of the development cycle local testing will be just as interesting and can be done on my development machine.

This project requires no additional file space on University machines. I will be hosting the project source code and dissertation files in a repository on github.\footnote{http://github.com/sebastian/Part-2-project}
If my machine breaks down, the development can be continued on any Unix based machine that has VIM, git and the Erlang VM installed.

\section*{Work Plan}

Planned starting date is 15/10/2010. 

Below follows a list of tasks that need to be done:

\begin{itemize}
  \item Work through the theory behind Chord, Kademilia and Pastry and other items listed under \emph{difficulties to overcome} (2 weeks).
  \item Implement initial version of Chord, Kademilia and Pastry in Erlang (4 weeks)
  \item Implement test harness to perform testing of the system (4 weeks)
  \item Implement a web frontend for search alongside a search server written in Erlang, that uses the distributed data storage layer for its data (2 weeks).
  \item Write dissertation (6 weeks).
\end{itemize}

\subsection*{Michaelmas Term} 

By the end of this term I intend to have completed the research and learning tasks and have finished the first implementations of the Distributed Hash Tables in Erlang. In the vacation that follows I intend to make a good start on the testing harness and do a little work on the search server.


\subsection*{Lent Term}

In the first half of this term I intend to finish the test harness and search server and spend time testing the system and solving problems. The tests could follow a factorial design where distributed hash table type, number of nodes, key-size, payload size, number of entries in the system and geographical distribution are all factors worth considering. What levels should be considered for each factor is yet to be determined and will become clearer as the project develops.

In the second half of this term I tend to get an initial draft of my dissertation written.


\subsection*{Easter Term}

In this term I plan to polish the dissertation. The estimated completion date is the 15th of May, leaving a couple of days to let the dissertation rest before giving it a final read and correcting last minute mistakes before the due date on the 20th of May.


\end{document}
