
% Draft #1 (final?)

\vfil

\centerline{\Large Friend search for distributed social networks }
\vspace{0.4in}
\centerline{\large Sebastian Probst Eide, St Edmunds College }
\vspace{0.3in}
\centerline{\large Originator: Sebastian Probst Eide}
\vspace{0.3in}
\centerline{\large 4 October 2010}

\vfil

\subsection*{Special Resources Required}
Machine with the Erlang VM installed \\
Virtual Servers running on Amazon EC2 and Cloudsigma \\
The use of my own laptop.
\vspace{0.2in}

\noindent
{\bf Project Supervisors:} Dr David Eyers and Dr David Evans
\vspace{0.2in}

\noindent
{\bf Director of Studies:} Dr Robert Harle
\vspace{0.2in}
\noindent
 
\noindent
{\bf Project Overseers:} Unknown 

\vfil
\pagebreak

% Main document

\section*{Introduction}

It is hard to get started using distributed independent social networks as it is almost impossible to find and connect with your existing friends, not knowing where and in which social network they host their profiles. The purpose of this project is to lay the foundation for a decentralized and distributed friend search engine that can be hosted alongside installations of independent social networks allowing users to easily reconstruct their social graph in the social network of their choice. The focus of the project will be on the data storage layer of the search engine, implementing, evaluating and comparing different Distributed Hash Tables implemented in Erlang. A front end, allowing basic searches to be performed, will also be created, but mainly to provide a way of rapidly testing the data storage layer. More user oriented functionality needed to allow the project to succeed on a larger scale will be left out to limit the scope of the project.

\section*{Work that has to be done}

There are a great number of distributed hash tables available\footnote{Wikipedia currently lists 8 different protocols}. I have decided to implement and compare the following three: Chord, Kademilia and Pastry. They all support the same basic operations of setting and retrieving key-value pairs, but use different heuristics in the way they route traffic, which makes it interesting to compare them against each other using amonst others, nodes that are geographically spread out.

The main parts of the project are to:

\begin{enumerate}
  \item Implement the data structure layer in Chord, Kademilia and Pastry using a uniform API that allows the system to use any one of the three without additional changes.
  
  \item Implement infrastructure that facilitates testing and monitoring of the system. More specifically it should allow:

  \begin{enumerate}
    \item Starting and stoping virtual servers accross the different service providers for testing purposes
    \item Start and stop search nodes accross the physical servers
    \item Display how many search nodes are available in the system, and how they are geographically distributed
    \item Add and remove test data from the system
    \item Perform repetable load testing on the system
    \item Count the number of jumps and time a key lookup needs to do to find a data item and calculate average and median times of lookups for fixed larger key-value datasets
    \item Being able to eliminate and add subsets of storage nodes in a repeateble fashion to test how the different Distributed Hash Tables cope with nodes dynamically dissapearing and appearing
  \end{enumerate}

  \item Setup Linux image that can be run accross Infrastructure as a Service providers\footnote{I will use Vagrant (http://vagrantup.com) and Chef for this}

  \item Implement web front end to allow users to perform basic searches accross data storage layer.

\end{enumerate}

Please note that the following aspects of the search server are secondary to the project and are not part of what will be implemented:

\begin{enumerate}
  \item Fuzzy searches allowing the user to misspell names
  \item Predictive searches
  \item Searches taking social circles or similar metadata into account
  \item Protecting against spam
\end{enumerate}


\section*{Starting Point}

I have a reasonable working knowledge of Erlang and linux and development of web based systems. The algorithms that will be implemented have all previously been implemented in other languages, and are used in production systems, so finding information about them should be possible. I have never used Amazon EC2 or any of the other Infrastructure as a Service (IaaS) providers I plan to use, but have used similar services in the past and don't see this as being a problem.


\section*{Success criterion}

I regard the project as successful if I have working implementations of the three Distributed Hash Tables that allow me to set and retrieve values based on keys accross a distributed network of machines. Additionally I should have metrics for how the performance of key-lookup varies by node count and distributed hash table type.

\section*{Difficulties to Overcome}

The following main learning tasks will have to be undertaken before 
the project can be started:

\begin{itemize}

\item To learn and fully understand the Chord, Kademilia and Pastry algorithms.

\item To learn how to do network communication in Erlang other than the built in message passing.

\item To learn how to use the OverSim\footnote{http://en.wikipedia.org/wiki/OverSim} simulation tool to test and verify my implementations of the Distributed Hash Tables.

\item To learn how to use Amazon EC2 and similar Infrastructure as a Service providers.

\end{itemize}


\section*{Resources}

Some aspects of this project (amongst other the heuristics in Pastry which take locality into account when routing) are more interesting to test in nodes that are in geographically distinct areas. For this reason I will use server instances running on commercial Infrastructure as a Service providers like Amazon AWS and Cloudsigma. For the majority of the development cycle, local testing will be just as interesting and can be done on my development machine.

This project requires no additional file space on University machines. I will be hosting the project source code and dissertation files in a repository on github\footnote{http://github.com/sebastian/Part-2-project}.
If my machine breaks down, the development can be continued on any Unix based machine that has the Erlang VM installed.

\section*{Work Plan}

Planned starting date is 15/10/2000. 

Below follows a list of tasks that need to be done:

\begin{enumerate}
  \item Work through the theory behind Chord, Kademilia and Pastry and other items listed under \emph{difficulties to overcome} (2 weeks).
  \item Implement initial version of Chord, Kademilia and Pastry in Erlang (4 weeks)
  \item Implement test harness to perform testing of the system (4 weeks)
  \item Implement search server using one of the Data Storage Layers implemented (2 weeks)
  \item Write dissertation (6 weeks).
\end{enumerate}

\subsection*{Michaelmas Term} 

By the end of this term I intend to have completed the research and learning tasks and have finished the first implementations of the Distributed Hash Tables in Erlang. In the vacation that follows I intend to get a good start on the testing harness and do a little work on the search server.


\subsection*{Lent Term}

In the first half of this term I intend to finish the test harness and search server and spend time testing the system and solving problems.

In the second half of this term I tend to get an initial draft of my dissertation written.


\subsection*{Easter Term}

In this term I plan to polish the dissertation. The estimated completion date is the 15th of May, leaving a couple of days to let the dissertation rest before giving it a final read and correcting last minute mistakes befor the due date on the 20th of May.
